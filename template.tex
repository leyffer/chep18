%%%%%%%%%%%%%%%%%%%%%%% file template.tex %%%%%%%%%%%%%%%%%%%%%%%%%
%
% This is a template file for Web of Conferences Journal
%
% Copy it to a new file with a new name and use it as the basis
% for your article
%
%%%%%%%%%%%%%%%%%%%%%%%%%% EDP Science %%%%%%%%%%%%%%%%%%%%%%%%%%%%
%
%%%\documentclass[option]{webofc}
%%% "twocolumn" for typesetting an article in two columns format (default one column)
%
\documentclass{webofc}
\usepackage[varg]{txfonts}   % Web of Conferences font
%
% Put here some packages required or/and some personnal commands
%
%

\def\kbdstyle{\smaller\bfseries\ttfamily}
\newcommand{\kbd}[1]{{\kbdstyle #1}\xspace}
\newcommand{\code}[1]{\textcolor{blue!60!green}{\kbd{#1}}\xspace}
\newcommand{\pkg}[1]{\textcolor{purple!40!black!80!white}{\kbd{#1}}\xspace}
\newcommand{\cmd}[1]{\kbd{#1}\xspace}
\newcommand{\subst}[1]{\ensuremath{\langle\text{\textit{#1}}\rangle}}

\renewcommand{\bold}[1]{\textbf{#1}}
\newcommand{\italic}[1]{\textit{#1}}
\newcommand{\highlight}[1]{\red{#1}\xspace}
\newcommand{\subdued}[1]{\grey{#1}\xspace}
\newcommand{\minor}[1]{\grey{\smaller #1}}

\newcommand{\col}[1]{\column{#1\textwidth}}

%%%%%%%%%%%%%%%%%%

\newcommand{\secref}[1]{Sect.~\ref{#1}}
\newcommand{\tabref}[1]{Tab.~\ref{#1}}
\newcommand{\figref}[1]{Fig.~\ref{#1}}
\newcommand{\mini}{\mathop{\mbox{minimize}}}
\newcommand{\maxi}{\mathop{\mbox{maximize}}}
\newcommand{\amin}{\mathop{\mbox{argmin}}}
\newcommand{\amax}{\mathop{\mbox{argmax}}}
\newcommand{\st}{\mbox{subject to }}
\newcommand{\bmath}[1]{\mbox{\boldmath $ #1 $}}
\newcommand{\R}{\mathbb{R}}
\newcommand{\dps}{\displaystyle}
\newcommand{\p}{\bmath{p}}
\newcommand{\w}{\bmath{w}}
\newcommand{\calS}{\mathcal{S}}
\newcommand{\calO}{\mathcal{O}}
%\newcommand{\p}{\ensuremath{\vec{p}\xspace}}
\newcommand{\phat}{\ensuremath{\hat{p}}\xspace}
\newcommand{\phisq}{\ensuremath{\phi^2}\xspace}
\newcommand{\phato}{\ensuremath{\hat{p}|\left\{w_\mathcal{O}\right\}\xspace}}
\newcommand{\pzero}{\ensuremath{\vec{p}_0}\xspace}
\newcommand{\pprime}{\ensuremath{\vec{p}'}\xspace}
\newcommand{\ptilde}{\ensuremath{\tilde{p}\xspace}}
\newcommand{\vptilde}{\ensuremath{\tilde{\vec{p}}}\xspace}
\renewcommand{\d}[1]{\ensuremath{\mathrm{d}#1}\xspace}
\newcommand{\nCr}[2]{\ensuremath{\parenths{\colvec{#1 \\ #2}}}\xspace}
\newcommand{\alphaS}{\ensuremath{\alpha_\text{s}}\xspace}
\newcommand{\alphaSMZ}{\ensuremath{\alpha_\text{s}(M_Z)}\xspace}
\newcommand{\pT}{\ensuremath{p_\mathrm{T}}\xspace}
\newcommand{\kT}{\ensuremath{k_\mathrm{T}}\xspace}

\newcommand{\qqbar}{\ensuremath{q\bar{q}}\xspace}
\newcommand{\QQbar}{\ensuremath{Q\bar{Q}}\xspace}
\newcommand{\ttbar}{\ensuremath{t\bar{t}}\xspace}
\newcommand{\bbbar}{\ensuremath{b\bar{b}}\xspace}
\newcommand{\ccbar}{\ensuremath{c\bar{c}}\xspace}
\newcommand{\ssbar}{\ensuremath{s\bar{s}}\xspace}

\newcommand{\tofrom}{\ensuremath{\leftrightarrow}}
\renewcommand{\to}{\ensuremath{\rightarrow}\xspace}
\newcommand{\To}{\ensuremath{\Rightarrow}\xspace}
\newcommand{\ToFrom}{\ensuremath{\Leftrightarrow}}
\newcommand{\tick}{\checkmark}

\newcommand{\Cxx}{C\!\,\raisebox{0.05\baselineskip}{++}\xspace}
\newcommand{\xx}{\!\,\raisebox{0.05\baselineskip}{++}\xspace}
\newcommand{\Fortran}{\textsc{Fortran}\xspace}

\newcommand{\eV}{\ensuremath{\text{e\kern-0.15ex{}V}}\xspace}
\newcommand{\MeV}{\ensuremath{\text{M\eV}}\xspace}
\newcommand{\GeV}{\ensuremath{\text{G\eV}}\xspace}
\newcommand{\TeV}{\ensuremath{\text{T\eV}}\xspace}

\begin{document}
%
\title{Teaching PROFESSOR new math}
%
% subtitle is optionnal
%
%%%\subtitle{Do you have a subtitle?\\ If so, write it here}

\author{
    \firstname{Anthony} \lastname{Austin}\inst{1}\fnsep\thanks{\email{apaustin@vt.edu}} \and
    \firstname{Sven} \lastname{Leyffer}\inst{2}\fnsep\thanks{\email{leyffer@mcs.anl.gov}} \and
    \firstname{Steven} \lastname{Mrenna}\inst{3}\fnsep\thanks{\email{mrenna@fnal.gov}} \and
    \firstname{Juliane} \lastname{M\"uller}\inst{4}\fnsep\thanks{\email{julianemueller@lbl.gov}} \and
    \firstname{Holger} \lastname{Schulz}\inst{3,5}\fnsep\thanks{\email{hschulz@fnal.gov}}
}

\institute{
           Virginia Tech \and
           Argonne National Laboratory \and
           Fermi National Accelerator Laboratory\and
           Lawrence Berkeley National Laboratory\and
            University of Cincinnati 
          }

\abstract{%
    The software package PROFESSOR~\cite{Buckley:2009bj} provides machinery to
    facilitate Monte-Carlo  event-generator tuning.  The method is based on
    the numerical optimization of a goodness-of-fit  measure defined between
    measured experimental data and MC predictions. As the latter are typically
    very expensive to evaluate, PROFESSOR replaces them with fast to evaluate
    surrogates. We present improvements to the method in the numerical
    optimization as well as the surrogate construction. Firstly, we introduce
    an algorithm and implementation of a bi-level optimization problem aimed at
    improving scenarios were the underlying physics model is unable to describe
    all data to the same accuracy. Secondly, we display our implementation of
    an algorithm for multivariate rational approximations which are superior to
    the previously used polynomial surrogates.
}
%
\maketitle

%
\section{Introduction}
\label{intro}
Monte-Carlo (MC) event generators are an essential tool for physics analyses. Their
predictions are used e.g. to estimate background contributions, to correct for
detector effects and ultimately test physics models against measured data. It
is therefore desirable to achieve particle simulations that are as precise as
possible. In the last decade, tremendous success was achieved in the
perturbative regime of the underlying calculations. Realistic MC generators,
however, also contain physics in a regime where couplings are large and field
theory methods break down.  Programs such as Pythia8~\cite{Sjostrand:2007gs}
implement non-perturbative aspects of the simulation by means of heuristic
models that introduce parameters that need to be adjusted to meaningful values
with the objective to reproduce nature with high fidelity.  This process, which
is commonly referred to as ``tuning'', is helped greatly by the existence of
tools like Rivet~\cite{Buckley:2010ar} which conveniently analyze the simulated
events to allow for immediate comparison with a trove of experimental data,
e.g.  in the form of a goodness-of-fit measure such as
\begin{equation}\label{eq:gof}
    \chi^2(p) = \sum\limits_b^{N_\text{bins}} \frac{\left[d_b - \text{MC}_b(p)\right]^2}{\sigma_b^2},
\end{equation}
where the sum runs over individual bins, $b$. The measured data is given by
$d_b$ and some uncertainty $\sigma_b$. The corresponding Monte-Carlo
prediction $\text{MC}_b(p)$ is dependent on model parameters, $p$.  The
optimization problem is defined such that we want to find the point $\hat{p}$
that minimizes the expression in \eqref{eq:gof}. The computational cost
of running the MC generators for a given point $p$ range from minutes
to days, depending on the physics process. For most cases, this is too
expensive for numerical minimization. PROFESSOR instead uses surrogates $f_b$
that are trained on a moderate number of $\text{MC}(p)$. The evaluation of
$f_b(p)$ is of the order of micro-seconds which is much more amenable for
numerical minimization purposes.

Ideally the models would be able to describe all measured
data with one set of parameter values. However, this is generally not the case. Instead, specialized tunes are developed by biasing the
optimization process to give more weight to certain datasets than others. This
process typically involves many iterations in a trial-and-error fashion. We
attempt to automate this procedure in \secref{sec-oopt} by formulating the problem in \eqref{eq:gof} as bilevel optimization problem.

Although fairly robust and widely applicable, the multivariate polynomial approximations
currently used  in PROFESSOR  do have limitations if the true functional form of the
to be approximated data exhibits traits of rational functions. In \secref{sec-rapp}
we demonstrate our implementation of an algorithm for multivariate rational
approximations.


\section{Bi-level Optimization}
\label{sec-oopt}

The difficulty with solving the ``classical'' tuning problem arise from the following challenges: 
    \begin{itemize}
        \item Incompatible datasets and mismodeling in the MC generator necessitate the introduction of tuning weights $w_\mathcal{O}$ for each observable
        \item Adjusting these weights has so far been a manual procedure in which the user iteratively optimizes  the classical problem (from here on ``inner optimization``) and look at resulting plots
      \end{itemize} 
       
To overcome these difficulties, we propose an automated procedure to replace the manual weight adjustment. For this purpose, we reformulate the problem as bilevel optimization problem, where on the upper level (``outer optimization'') we optimize over the weights-per-observable, and on the lower level (``inner optimization'') we optimize over the parameters $p$ such that the goodness of fit function is minimized given the weights from the upper level.

The resulting bilevel optimization problem then becomes:
\begin{subequations}
\begin{equation}
\min_{\w \in[0,1]^{N_\calO}, \hat{\p}_{\w}\in\Omega} g(\w, \hat{\p}_{\w})\label{eq:outer_gen}
\end{equation}
\begin{equation}
\text{s.t. } \sum_{\calO=1}^{N_{\calO}} w_\calO=1\label{eq:sum1}
\end{equation}
\begin{equation}
\hat{\p}_{\w} = \arg\min_{\p\in\Omega} \phi^2(\p, \w) 
\label{eq:inner}\end{equation}
\end{subequations}
where $\w$ represents the vector of weights per observable, $N_\calO$ is the number of observables, $\Omega$ is the space in which the parameters $\p$ live, and $\hat{\p}_{\w}$ is the optimal solution (the optimal parameters) as determined by solving the inner optimization problem for a given set of weights $\w$. The function $g(\cdot)$ represents our objective function at the outer level (more details below). Thus, in order to compute the objective function at the upper level \eqref{eq:outer_gen}, we have to solve the optimization problem at the inner level \eqref{eq:inner}, which is defined as
%            \begin{itemize}
%                \item Write goodness-of-fit in terms of histograms/observables
%                \item The parameter space is now the observable-weight space
%                \item Inner optimization yields best fit point, \phat, for given $$\left\{w_\mathcal{O}\right\}$$
%                \item \phat is used to evaluate an objective function for the outer optimization
%            \end{itemize}

\begin{equation}\label{eq:inneropt}
    \phisq(\p,\w) =
    \sum_{\mathcal{O}=1}^{N_\calO} w_\mathcal{O}^2 \cdot
    \sum_{b\in \mathcal{O}}
  \frac{ (f_b(\p) - d_b)^2 }{   \sigma_b^2}.
\end{equation}

For each $\hat{\p}_{\w}$, we can calculate the per-observable goodness-of fit
    
\begin{equation}
    \nu_\mathcal{O}(\hat{\p}_{\w}) =
    \frac{1}{N_\text{bins}(\mathcal{O})}
    \sum_{b\in \mathcal{O}}
  \frac{ (f_b(\hat{\p}) - d_b)^2 }{  \sigma^2}, \ {\cal O} =1,\ldots, N_{\calO}
\end{equation}

With $N_{\calO}$ such measures, we can calculate the mean and the standard deviation over the goodness of fit values of all observables:

\begin{equation} 
 \mu(\hat{\p}_{\w}) = \frac{1}{N_{\calO}}\sum_{{\cal O}=1}^{N_{\calO}} \nu_{\cal O}(\hat{\p}_{\w})
\end{equation}	
and
\begin{equation} 
s^2(\hat{\p}_{\w}) = \frac{1}{N_{\calO}}\sum_{{\cal O}=1}^{N_\calO} \left[\nu_{\cal O}(\hat{\p}_{\w}) -\mu(\hat{\p}_{\w})\right]^2.
\end{equation}	
We then define the objective function of the outer optimization problem as: 


\begin{equation}\label{eq:portf}
g(\w, \hat{\p}_{\w}) = 
\mu(\hat{\p}_{\w}) +\lambda s^2(\hat{\p}_{\w}),
\end{equation}	
where the dependency on the weights $\w$ arises from determining $\hat{\p}_{\w}$.

The minimization of the outer objective in \eqref{eq:portf} is done utilizing a radial basis function (RBF) which after a training phase will iteratively suggest new points in the weight-space until certain termination criteria of the algorithm are met. We notice that the results have a dependence on the initial guess, which is why we apply a multi-start strategy. Although not ideal this is certainly defensible in the light of the fast evaluation of the surrogates and therefore quick convergence of the inner optimization (\eqref{eq:inneropt}). In figure~\ref{fig-ooevol} we investigate the behaviour of the \emph{outer} optimization, showing the portfolio objective as a function of the number of iterations in the algorithm. We typically find the program to terminate after $\mathcal{O}(100)$ iterations per initial set of weights.

\begin{figure}[h]
% Use the relevant command for your figure-insertion program
% to insert the figure file.
\centering
    \begin{minipage}{.48\textwidth}%
        \begin{center}
            \includegraphics[width=.98\textwidth]{oo/chain-3-report.pdf}
        \end{center}
    \end{minipage}%
    \begin{minipage}{.04\textwidth}%
    \end{minipage}%
    \begin{minipage}{.48\textwidth}%
        \begin{center}
            \includegraphics[width=.98\textwidth]{oo/chain-1-report.pdf}
        \end{center}
           
    \end{minipage}%
\caption{Performance of the radial-basis function bases optimization for two different initial conditions. The training phases is the shaded area, the dashed horizontal line indicates the minimum objective found. The blue line shows the smallest known objective after each iteration.}
\label{fig-ooevol}       % Give a unique label
\end{figure}


\subsection*{Example}

As a first example, we are looking at a typical tuning problem, namely the so-called underlying event (UE).
This effect, which is present at any hadron-collider data is modelled using various parameters, steering such things as the hadronic matter-overlap, an energy evolution of the phenomenon as well as a necessary low-energy cut-off. Many dedicated measurements exist that are designed to be particularly sensitive to UE physic modelling. We have chosen the data presented in ~\cite{Aad:2011qe}. The modelling in Pythia8, which we shall use as our physics simulator, when compared to the data shows some level of discrepancy. Data at different collider energies can not necessarily be described with the same modelling parameters. Further, there is some tension between the number and energy of particles generated in the simulation. In the plots shown in figure~\ref{fig-oocmp} we compare our results obtained with the bi-level optimization to the standard tuning approach. We would argue that in the presence of mis-modelling, the bi-level optimization is able to yield a balanced prediction without human intervention.

 

\begin{figure}[!h]
\centering
    \begin{minipage}{.48\textwidth}%
        \begin{center}
            \includegraphics[width=.92\textwidth]{oo/test_cmp_1_std/ATLAS_2011_S8994773/d01-x01-y01.pdf}\\
            \includegraphics[width=.92\textwidth]{oo/test_cmp_1_std/ATLAS_2011_S8994773/d03-x01-y01.pdf}
        \end{center}
    \end{minipage}%
    \begin{minipage}{.04\textwidth}%
    \end{minipage}%
    \begin{minipage}{.48\textwidth}%
        \begin{center}
            \includegraphics[width=.92\textwidth]{oo/test_cmp_1_std/ATLAS_2011_S8994773/d02-x01-y01.pdf}\\
            \includegraphics[width=.92\textwidth]{oo/test_cmp_1_std/ATLAS_2011_S8994773/d04-x01-y01.pdf}
        \end{center}
    \end{minipage}%
    \caption{Comparison of results obtained with the bi-level optimization and the standard approach. The physics modelling in Pythia8 is such that there is some level of incompatibility from left to right as well as top to bottom in those 4 plots, meaning that one can get a very good description for each of the distributions alone --- but not together. We would argue that the bi-level optimization with the portfolio objective is able to yield a more balanced prediction of these data-sets.}
\label{fig-oocmp} 

\end{figure}






\section{Rational approximations}
\label{sec-rapp}

Rational approximations can be seen as natural extensions of the polynomial
approximations used so far in PROFESSOR. Denoting the degrees of the numerator 
and denominator polynomial as $m$ and $n$ respectively, we can formally define
the rational approximation as
\begin{equation}\label{eq:rappdef}
    f^{(m,n)}(p) = \frac{h^{(m)}(p)}{g^{(n)}(p)}.
\end{equation}
Polynomial approximations in this picture are thus simply the class of rational approximations with $n=0$.

\subsection{The algorithm}

For simplicity, we present the algorithm for a one-dimensional parameter space. We are interested in deriving a rational approximation:
\begin{equation}\label{eq:PadeOneD}
  f : \R \to \R, \quad f(p) \simeq \frac{h(p)}{g(p)}
\end{equation}
by fitting the coefficients of $h(p), g(p)$ to data, $(p_l, f(p_l)$. Denoting the polynomials
as
\begin{equation}
  h(p) = a_0 + a_1 p + a_2 p^2 + \ldots + a_m p^m \quad \text{and} \quad
  g(p) = b_0 + b_1 p + b_2 p^2 + \ldots + b_n p^n,
\end{equation}
we observe that we have $K=(m+1)+(n+1)-1$ degrees of freedom (one less, because we can scale
either $a_0=1$ or $b_0=1$). Assuming that we have $K$ data points, $\left(p_l, f(p_l)\right), l=1,\ldots,K$
that are ``nicely'' chosen, the fitting problem can be written as
\begin{equation}\label{eq:FitPadeOneD}
  f(p_l) =  \frac{h(p_l)}{g(p_l)} \quad \Leftrightarrow \quad g(p_l)f(p_l) =  h(p_l) \quad \forall l=1,\ldots,K
\end{equation}
provided that $g(p_l) \not = 0$. We note, that \eqref{eq:FitPadeOneD} is a square linear
system of equations in $K$ unknowns. To solve this system, we provisionally add coefficients
up to order $K$ to $h(p)$ and $g(p)$ (which we will enforce to be zero later). Thus, we write
the polynomials as
\begin{equation}
  \begin{array}{rcl}
    h(p) & = & a_0 + a_1 p + a_2 p^2 + \ldots + a_m p^m + \ldots + a_K p^K \\
    \text{and} & & \\
    g(p) & = & b_0 + b_1 p + b_2 p^2 + \ldots + b_n p^n + \ldots + b_K p^K.
  \end{array}
\end{equation}
We let the Vandermonde matrix of order $K$ be
\begin{equation}
  V := \left[ \begin{array}{ccccc}
      1 & p_1 & p_1^2 & \ldots & p_1^K \\
        &     &      &        &       \\
      \vdots & \vdots & \vdots &   & \vdots \\
        &     &      &        &       \\
      1 & p_K & p_K^2 & \ldots & p_K^K 
       \end{array}\right]
\end{equation}
and we define the diagonal matrix $F := \text{diag}(f(p_1), \ldots, f(x_K))$, and
vectors of coefficients $a=(a_0,\ldots,a_K)^T$ and $b=(b_0,\ldots,b_K)^T$. Then,
we can write \eqref{eq:FitPadeOneD} compactly as
\[ F V b = V a . \]
If we assume that $V$ is invertible (which imposes a condition on the interpolation points),
then we can rewrite this system as
\[ a = V^{-1} F V b =: Z b , \]
where $Z = V^{-1} F V$. Now recall, that $b_{n+1} = \ldots = b_K = 0$, and denote the first
$n$ components of $b$ by $\hat{b}$, and similarly define $\hat{Z} := Z[:,1:n]$ using Matlab/Python
notation. Then it follows that
\[ a = \hat{Z} \hat{b} ,\]
which is a ``skinny'' system of equations. If we also enforce the condition that
$a_{m+1} = \ldots = a_K = 0$, then we can write this system as
\[ \left(\begin{array}{c}
     a_0 \\ \vdots \\ a_m \\ \hline a_{m+1} \\ \vdots \\ a_K
   \end{array}\right)
   =
   \left[\begin{array}{c} \\ \bar{Z} \\ \\ \hline \\ \tilde{Z} \\ \phantom{+} \end{array}\right]
   \left(\begin{array}{c}
     b_0 \\ \vdots \\ b_n
   \end{array}\right)
   \quad \text{or}\quad \hat{a} = \bar{Z} \hat{b}, \quad \text{and} \quad 0 = \tilde{Z} \hat{b}
\]
We now observe that the last set of equations, $0 = \tilde{Z} \hat{b}$, has $K-m-1=m+n+1-m-1=n$ equations
and $n+1$ unknowns. Thus, if the equations are consistent, we can choose $\hat{b}$ to be a vector that
lies in the null-space of $\tilde{Z}$. Forming an SVD of $\tilde{Z}$
\[ \tilde{Z} = \tilde{U} \tilde{\Sigma} \tilde{V}^T ,\]
we choose $\hat{b} := \tilde{V}[:,n+1]$ as the last singular vector, then obtain $\hat{a} = \bar{Z} \hat{b}$.

The generalization to multi-dimensional parameter spaces is straight forward.


\subsection{Example}

We consider the function $$f(x) = \frac{1 + x}{1 + x + x^2},\quad
x\in(0,100).$$ The calculation of a rational approximation of type $(m=1, n=2)$
requires the determination of $N_\text{coeff}=6$ coefficients, which is the
same number of coefficients required to build a $5^\text{th}$ order polynomial
in 1 dimension.  Both approximations to the same input data are shown in
\figref{fig:cmp}. The qualitative gain when using rational approximations is
quite evident. Rather importantly for physics use-cases, where purely positive
definite quantities are encountered frequently, we observe that the polynomial
approximations have a tendency to oscillate, leading to negative predictions
which can be nonphysical therefore making the application of the latter at least
questionable.

\begin{figure}[!h]
    \begin{minipage}{.48\textwidth}%
        \begin{center}
            \includegraphics[width=.98\textwidth]{code/test04_01_2.pdf}
        \end{center}
    \end{minipage}%
    \begin{minipage}{.04\textwidth}%
    \end{minipage}%
    \begin{minipage}{.48\textwidth}%
        \begin{center}
            \includegraphics[width=.98\textwidth]{code/test04_05_0.pdf}
        \end{center}
    \end{minipage}%
    \centering
    \caption{Comparison of rational and polynomial approximation to input data
    that exhibits traits of a rational function. Left: rational approximation
with 6 coefficients. Right: polynomial approximation with 6 coefficients.}
    \label{fig:cmp}
\end{figure}


%In higher dimensional parameter spaces it is useful to look at the difference between predicted function values and known function values at dedicated test points.





%\input{dummy.tex}

\newpage
\section{Summary and outlook}
We presented recent progress in the methodology of of Monte-Carlo event-generator tuning. A bi-level optimization strategy seems promising toward more automatic event-generator tuning, especially in the presence of mis-modelling or otherwise existing tension between data-sets. The extension of the surrogate models from ordinary polynomials to rational approximations looks promising and is expected to widen the applicability of the tuning method. The application to multi-dimensional parameter spaces will be thoroughly discussed in an upcoming publication.

%
% BibTeX or Biber users please use (the style is already called in the class, ensure that the "woc.bst" style is in your local directory)
 \bibliography{references.bib}
\section*{Acknowledgement}
 This material is based upon work supported by the U.S. Department of Energy,
 Office of Science, Office of Advanced Scientific Computing Research and Office
 of HEP, Scientific Discovery through Advanced Computing (SciDAC) program.


\end{document}

% end of file template.tex
